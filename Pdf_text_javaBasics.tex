\documentclass[]{article}
\usepackage{lmodern}
\usepackage{amssymb,amsmath}
\usepackage{ifxetex,ifluatex}
\usepackage{fixltx2e} % provides \textsubscript
\ifnum 0\ifxetex 1\fi\ifluatex 1\fi=0 % if pdftex
  \usepackage[T1]{fontenc}
  \usepackage[utf8]{inputenc}
\else % if luatex or xelatex
  \ifxetex
    \usepackage{mathspec}
  \else
    \usepackage{fontspec}
  \fi
  \defaultfontfeatures{Ligatures=TeX,Scale=MatchLowercase}
\fi
% use upquote if available, for straight quotes in verbatim environments
\IfFileExists{upquote.sty}{\usepackage{upquote}}{}
% use microtype if available
\IfFileExists{microtype.sty}{%
\usepackage{microtype}
\UseMicrotypeSet[protrusion]{basicmath} % disable protrusion for tt fonts
}{}
\usepackage[margin=1in]{geometry}
\usepackage{hyperref}
\hypersetup{unicode=true,
            pdftitle={Pdf\_text},
            pdfauthor={Vijay Mudivedu},
            pdfborder={0 0 0},
            breaklinks=true}
\urlstyle{same}  % don't use monospace font for urls
\usepackage{graphicx,grffile}
\makeatletter
\def\maxwidth{\ifdim\Gin@nat@width>\linewidth\linewidth\else\Gin@nat@width\fi}
\def\maxheight{\ifdim\Gin@nat@height>\textheight\textheight\else\Gin@nat@height\fi}
\makeatother
% Scale images if necessary, so that they will not overflow the page
% margins by default, and it is still possible to overwrite the defaults
% using explicit options in \includegraphics[width, height, ...]{}
\setkeys{Gin}{width=\maxwidth,height=\maxheight,keepaspectratio}
\IfFileExists{parskip.sty}{%
\usepackage{parskip}
}{% else
\setlength{\parindent}{0pt}
\setlength{\parskip}{6pt plus 2pt minus 1pt}
}
\setlength{\emergencystretch}{3em}  % prevent overfull lines
\providecommand{\tightlist}{%
  \setlength{\itemsep}{0pt}\setlength{\parskip}{0pt}}
\setcounter{secnumdepth}{0}
% Redefines (sub)paragraphs to behave more like sections
\ifx\paragraph\undefined\else
\let\oldparagraph\paragraph
\renewcommand{\paragraph}[1]{\oldparagraph{#1}\mbox{}}
\fi
\ifx\subparagraph\undefined\else
\let\oldsubparagraph\subparagraph
\renewcommand{\subparagraph}[1]{\oldsubparagraph{#1}\mbox{}}
\fi

%%% Use protect on footnotes to avoid problems with footnotes in titles
\let\rmarkdownfootnote\footnote%
\def\footnote{\protect\rmarkdownfootnote}

%%% Change title format to be more compact
\usepackage{titling}

% Create subtitle command for use in maketitle
\newcommand{\subtitle}[1]{
  \posttitle{
    \begin{center}\large#1\end{center}
    }
}

\setlength{\droptitle}{-2em}
  \title{Pdf\_text}
  \pretitle{\vspace{\droptitle}\centering\huge}
  \posttitle{\par}
  \author{Vijay Mudivedu}
  \preauthor{\centering\large\emph}
  \postauthor{\par}
  \predate{\centering\large\emph}
  \postdate{\par}
  \date{2018-06-11}


\begin{document}
\maketitle

\subsection{PDF to text}\label{pdf-to-text}

\begin{verbatim}
## Loading required package: RColorBrewer
\end{verbatim}

\begin{verbatim}
## -- Attaching packages -------------------------------------------------------------------------- tidyverse 1.2.1 --
\end{verbatim}

\begin{verbatim}
## √ ggplot2 2.2.1     √ purrr   0.2.4
## √ tibble  1.4.2     √ dplyr   0.7.4
## √ tidyr   0.8.0     √ stringr 1.3.0
## √ readr   1.1.1     √ forcats 0.3.0
\end{verbatim}

\begin{verbatim}
## -- Conflicts ----------------------------------------------------------------------------- tidyverse_conflicts() --
## x dplyr::filter() masks stats::filter()
## x dplyr::lag()    masks stats::lag()
\end{verbatim}

\begin{verbatim}
## Loading required package: NLP
\end{verbatim}

\begin{verbatim}
## 
## Attaching package: 'NLP'
\end{verbatim}

\begin{verbatim}
## The following object is masked from 'package:ggplot2':
## 
##     annotate
\end{verbatim}

\begin{itemize}
\tightlist
\item
  Download the data from google drive
\end{itemize}

\begin{verbatim}
## File downloaded:
##   * JavaBasics-notes.pdf
## Saved locally as:
##   * JavaBasics-notes.pdf
\end{verbatim}

\begin{itemize}
\tightlist
\item
  Store the data in the
\end{itemize}

\begin{verbatim}
##  chr [1:23] "                                                                                         Java Basics\nJava Basi"| __truncated__ ...
\end{verbatim}

\subsubsection{Data Cleaning}\label{data-cleaning}

\begin{itemize}
\item
  Remove the ``Java Basics'' header, special characters, the Footers
\item
  Combining all the text from different pages.
\end{itemize}

\begin{verbatim}
## <<SimpleCorpus>>
## Metadata:  corpus specific: 1, document level (indexed): 0
## Content:  documents: 1
\end{verbatim}

\paragraph{Further Cleaning of text}\label{further-cleaning-of-text}

\begin{itemize}
\item
  Lower case
\item
  Removing punctuations
\item
  Removing white spaces
\item
  Removing numbers
\item
  Removing stopwords
\item
  Creating Document term-matirx for the cleaned text
\item
  Most frequent terms are:
\end{itemize}

\begin{verbatim}
##  [1] "allocate"  "applet"    "applets"   "array"     "browser"  
##  [6] "button"    "can"       "class"     "code"      "comments" 
## [11] "data"      "example"   "following" "garbage"   "int"      
## [16] "java"      "language"  "may"       "memory"    "method"   
## [21] "new"       "null"      "object"    "objects"   "pointer"  
## [26] "primitive" "program"   "public"    "return"    "runtime"  
## [31] "stack"     "string"    "types"     "use"       "void"
\end{verbatim}

\begin{itemize}
\tightlist
\item
  Frequency of the most frequent terms
\end{itemize}

\begin{verbatim}
##         java          new         data         code          int 
##           75           52           39           38           34 
##       applet        class       button        array      objects 
##           30           29           28           25           24 
##       method       object       public      example       string 
##           23           23           22           19           19 
##     language       memory         null       return        types 
##           15           15           15           15           14 
##          use         void     comments      browser          may 
##           14           14           13           12           12 
##    primitive      program     allocate          can      applets 
##           12           12           11           11           10 
##    following      garbage      pointer      runtime        stack 
##           10           10           10           10           10 
##        adata applications       called         file         make 
##            9            9            9            9            9 
##      methods     operator    reference       system          two 
##            9            9            9            9            9 
##         byte   collection         init     programs         stat 
##            8            8            8            8            8
\end{verbatim}

\begin{itemize}
\tightlist
\item
  creating the word dataframe
\end{itemize}

\begin{verbatim}
##    words freq
## 1   java   75
## 2    new   52
## 3   data   39
## 4   code   38
## 5    int   34
## 6 applet   30
\end{verbatim}

\begin{itemize}
\tightlist
\item
  Printing the Keywords In Java\_basic in the text file.
  \includegraphics{Pdf_text_javaBasics_files/figure-latex/unnamed-chunk-14-1.pdf}
\end{itemize}


\end{document}
